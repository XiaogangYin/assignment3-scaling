\titledquestion{Scaling Laws Review}[5] 
\begin{parts}

    
    \part[5] Problem (chinchilla\_isoflops)\newline
        \quad\quad Write a script to reproduce the IsoFLOPs method describe above for fitting scaling laws using
        the final training loss from a set of training runs. For this problem, use the (synthetic) data from
        training runs given in the file data/isoflops\_curves.json. This file contains a JSON array, where
        each element is an object describing a training run. Here are the first two runs for illustrating the
        format:
        \begin{lstlisting}[language=Python]
            [
                {
                "parameters": 49999999,
                "compute_budget": 6e+18,
                "final_loss": 7.192784500319437
                },
                {
                "parameters": 78730505,
                "compute_budget": 6e+18,
                "final_loss": 6.750171320661809
                },
                ...
            ]
        \end{lstlisting}
        For fitting the scaling laws, the scipy package (and scipy.optimize.curve\_fit in particular)
        might be useful, but you’re welcome to use any curve fitting method you’d like. While Hoffmann et al.
        [2022] fits a quadratic function to each IsoFLOP profile to find its minimum, we instead recommend
        you simply take the run with the lowest training loss for each compute budget as the minimum.
        \begin{subparts}

            \subpart Show your extrapolated compute-optimal model size, together with the $\langle C_i,N_{opt}(C_i) \rangle$ points you
            obtained. What is your predicted optimal model size for a budget of $10^{23}$ FLOPs? What about for $10^{24}$ FLOPs?

            \textbf{Deliverable}: A plot showing your scaling law for model size by compute budget, showing the
            data points used to fit the scaling law and extrapolating up to at least $10^{24}$ FLOPs. Then, a
            one-sentence response with your predicted optimal model size.

            \ifans{
            My predicted optimal model size for a budget of $10^{23}$ FLOPs is 39,792,556,129, and 85,491,178,611 for $10^{24}$ FLOPs.
            
            I have obtained the following formula using scipy.optimize.curve\_fit.
                \begin{align*}
                N &= a \cdot C^b + c  &&\text{a=1.350e+03, b=3.253e-01, c=-1.192e+09} \\
                  &= 1350 \cdot C^{0.3253} - 1.192 \cdot 10^9
                \end{align*}
        
            \begin{figure}[H]  
                \centering  
                \includegraphics[width=0.5\textwidth]{C_N.pdf}  % 核心命令
                \caption{C-N figure}  % 图片标题(自动编号)
                \label{fig:c-n}  % 标签,用于后文引用(如\ref{fig:sin_plot})
            \end{figure}
            }
                   
            \subpart Show your extrapolated compute-optimal dataset size, together with the $\langle C_i,D_{opt}(C_i) \rangle$ data
            points from the training runs. What is your predicted optimal dataset size for budgets of $10^{23}$
            and $10^{24}$ FLOPs?
            \textbf{Deliverable}: A plot showing your scaling law for dataset size by compute budget, showing the
            data points used to fit the scaling law and extrapolating up to at least $10^{24}$ FLOPs. Then, a
            one-sentence response with your predicted optimal dataset size.
                
            \ifans{
            My predicted optimal model size for a budget of $10^{23}$ FLOPs is  364,796,370,806, and 1,525,768,382,027 for $10^{24}$ FLOPs.
            
            I have obtained the following formula using scipy.optimize.curve\_fit.
                \begin{align*}
                D &= a \cdot C^b + c  &&\text{a=1.802e-03, b=6.220e-01, c=5.948e+08} \\
                  &= 0.001802 \cdot C^{0.6220} + 5.948 \cdot 10^8
                \end{align*}
        
            \begin{figure}[H]  
                \centering  
                \includegraphics[width=0.5\textwidth]{C_D.pdf}  % 核心命令
                \caption{C-D figure}  % 图片标题(自动编号)
                \label{fig:c-d}  % 标签,用于后文引用(如\ref{fig:sin_plot})
            \end{figure}
            }

            \end{subparts}
        
        
            

\end{parts}
